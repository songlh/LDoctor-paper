Writing efficient software is difficult.
Design and implementation defects can easily
lead to severe performance 
degradation. Techniques that help
diagnose performance problems are desired.
Unfortunately, existing techniques are still preliminary.
%failing to provide the desired diagnosis coverage, accuracy, and performance. 
Performance-bug detectors can identify specific type
of inefficient computation, but are not suitable for accurately diagnosing
a wide variety of real-world performance problems. 
Profilers can locate code regions that consume resources, but 
not the ones that \textit{waste} resources.
Statistical performance diagnosis can
identify loops or branches that are most correlated with a performance
symptom, but cannot decide whether and why a loop is inefficient, and
how developers might fix it.

In this paper, we first design a root-cause
and fix-strategy taxonomy for real-world inefficient loops, 
one of the most common performance problems in the field.
We then design a static-dynamic hybrid analysis tool, \Tool, to
provide accurate performance diagnosis for loops.
We further use sampling techniques to lower the run-time overhead without
degrading the accuracy or latency of \Tool diagnosis. 
Evaluation using real-world performance
problems shows that \Tool can provide better coverage and accuracy than
existing techniques, with low overhead.
