\section{Evaluation of \Tool}
\label{sec:experiment}

\subsection{Methodology}
\label{sec:result_meth}
%Please discuss the potential usage scenarios of \Tool
%How you will use it together with other tools 

\input section/tab_benchmarks

\noindent\textbf{Implementation and Platform}
We implement \Tool in LLVM-3.4.2 \cite{llvm}, and conduct our
%Our implementation consists of 17654 lines of C++ code, 
%with 4748 for instrumenter, 674 for resultless analysis, 
%5802 for redundancy analysis, 
%and the remaining 6430 for common utilities. 
experiments on an i5-3330S machine, with Linux 3.19 kernel. 

\noindent\textbf{Benchmarks}
\Tool is a diagnosis tool that helps understand and fix 
already-manifested performance problems.
Consequently, our benchmarks are performance problems that have already
happened in real world.
For a thorough evaluation, we use
benchmarks from two different sources.

First, \emph{general} benchmark suite.
We evaluate \Tool on all bugs, 18 in total, that we can reproduce 
among the 45 bugs discussed in Section \ref{sec:eval_taxonomy}. 
These 18 bugs cover a wide variety of inefficiency root causes, as 
shown in the upper half of Table~\ref{tab:benchmarks}. 
We use the problem-triggering inputs reported by real-world
users in \Tool run-time analysis.
Among these 18, seven are extracted from Java/JavaScript
programs.
%, as \Tool currently only handles C/C++
%programs; one is extracted from a very old version of Mozilla that can no longer be
%compiled as it is.
%These 18 bugs include all the benchmarks used in the recent
%statistical performance debugging work~\cite{SongOOPSLA2014}.
%The extraction is conducted by XXX.
For these benchmarks, we did our best to keep all bug-related data structures
and caller/callee functions intact, and re-implement them in C++
following the original data and control flow. 
%Their
%bug-triggering inputs are produced based on the bug-triggering inputs in 
%the original bug reports.
We expect \Tool to also work for most of 
the remaining 27 bugs in Table \ref{tab:root}.
However, these 27 bugs are too difficult to reproduce, as they 
depend on special environment (e.g., Windows OS and 
.Net libraries) and cannot be easily extracted or reimplemented. 
For example, some bugs are related to GUI widgets and some 
require big complicated graphs to trigger.
%, which are too complicated to 
%re-implement. Some bugs' inefficient 
%loops traverse big graphs, whose bug-triggering inputs
%(i.e., the graphs) are difficult to figure out 
%without reproducing the original bugs. 

%Second, we evaluate \Tool on 21 out of the 42 detected by Toddler \cite{Alabama, toddlerbuglist}.
%{\color{red} We use all benchmarks that are reported with bug-triggering inputs and fixed by developers. 
%All of these bugs come from Java programs. 
%We extract XXX (how did you
%extract exactly?) and re-implement in C++. Each extracted benchmark 
%contains 4--5 loops (how come you always have 4--5 loops??XXX). 
Second, we evaluate \Tool on Toddler benchmark suite~\cite{Alabama, toddlerbuglist}.
Toddler project provides the bug-triggering inputs and detailed explanation
for 21 inefficient-loop bugs that have been confirmed and fixed by developers, and
we evaluate \Tool on \emph{all} of them.
Due to the focus of Toddler, all of these bugs are caused by inefficient
\textit{nested} loops and only cover two types of inefficiency root causes,
as shown in the bottom half of Table \ref{tab:benchmarks}.
We extract these bugs and re-implemented them in 
C/C++;
we re-implement basic Java data structures following a recent version of 
\texttt{openjdk}. Each extracted benchmark contains at least five
loops, except for two cases with four loops each.
%{\color{red} All re-implemented benchmarks used in our experiments are released on github~\cite{re-implemented}.}

The general benchmark suite and Toddler benchmark suite both provide
a large set of repeatable inefficient loop problems that we can access. 
At the same time, they were initially set up for different reasons and
methodologies,
and hence well complement each other.
All the benchmarks led to severe performance problems.
After developers fixed these problems, the performance of each benchmark improves
4X -- 500X under the user-reported workload in our experiments. 


\noindent\textbf{Evaluation settings}
Our evaluation uses existing statistical performance diagnosis
tool \cite{SongOOPSLA2014} to process a performance problem and identify 
a few suspicious loops for \Tool to analyze.
For all but four benchmarks, statistical debugging identifies the
real root-cause loop as the most suspicious loop. For the remaining four,
which all come from Table \ref{tab:root},
the real root-cause loops are ranked number 2, 2, 4, and 10.
%Overall, we believe future tools can accurately identify the most one or a couple
%of suspicious loops.
%XXX (what can we say about Toddler benchmark set here?)
%\input section/tab_cover
We use 0.001 as the default \textit{resultful rate} threshold for identifying
0*1? 
%loops, 0.01 as the \textit{resultful rate} threshold for identifying 
and [0$|$1]* resultless loops; we use 
0.5 as the default \textit{redundancy rate} threshold for identifying redundant loops.
%, and 
%2 as the \textit{cross-iteration redundancy rate} (i.e., 
%the number of distinct iterations is less than half of the total iterations).
We use $1/100$ sampling rate for cross-loop redundancy analysis and 
$1/1000$ sampling rate for cross-iteration redundancy analysis
(there tend to be more loop iterations than loop instances).
All our diagnosis results require only \textbf{one} run under the 
problem-triggering input.
All our performance results are obtained by taking average of \textbf{ten} runs, 
with the variation among different runs always less than 5\%.

%More discussions about all the parameters/thresholds presented above, including
%how to set them and how sensitive they are, are discussed in Section
%\ref{sec:sensi}. 

{\color{red}
\noindent\textbf{Research questions to answer}
We mainly conduct three sets of evaluation to answer three key research 
questions regarding the coverage, accuracy, and performance of \Tool, 

\textbf{RQ1}.
Can \Tool correctly identify inefficiency categories and fix 
strategies for many real-world loops? (Sec.\ref{sec:coverage})

\textbf{RQ2}.
Might \Tool report many false positives and hence waste developers' diagnosis
effort? (Sec.\ref{sec:result_acc})

\textbf{RQ3}.
What is the run-time overhead of \Tool? (Sec.\ref{sec:result_perf})
}


\subsection{Coverage Results}
\label{sec:coverage}
{\color{red}
To answer \textbf{RQ1}, we apply \Tool to real root-cause loops to see if 
it can accurately identify inefficiency categories.
As shown by the check-marks in Table~\ref{tab:benchmarks}
(the ``Did \Tool Identify ..?'' columns), 
\Tool does provide good diagnosis coverage.
}
\Tool identifies the correct root cause for \textbf{all} \allbugs benchmarks, and 
suggests fix strategies that match what developers took in practice
for 33 out of \allbugs cases. 

The six cases where \Tool and developers suggest/take different fix strategies
fall into three categories.
First, the fix strategy taken by developers is a subset of what suggested by 
\Tool.
For MySQL\#27287 and Apache\#53622, the root-cause loops contain both
cross-loop redundancy and 0*1? inefficiency. Consequently, \Tool suggests two
fix strategies. In practice, the developers
acknowledge both types of inefficiencies, but the patches
only changed the data structures, which eliminated both types of inefficiencies 
in case of MySQL\#27287 and left the cross-loop redundancy unsolved in
Apache\#53622. 
%For Apache\#53622, there are two buggy loops, one is cross-loop redundant, and the other is 0*1? inefficient. 
%\Tool suggests batching and changing data structure, respectively. 
%In practice, developers only fix the main root cause loop, by using changing data structure.} 

Second, the fix strategy taken by developers is related to
what suggested by \Tool.
The root-cause loops in 
Collections bugs \#407, \#408, and \#425 all conduct frequent linear searches in arrays.
\Tool suggests data-structure changes for these three cases. Developers' patches 
still keep the original data structures, but they
did use hash-sets, which contain the same content as the arrays,
to help conditionally skip the loops. 

Third, \Tool cannot suggest fix strategy for 1* loops.
For GCC\#12322, \Tool correctly tells that the loop under study
does not contain any form of inefficiency and produce results in every 
iteration, and hence fails to suggest any fix strategy. In practice, GCC
developers decide to skip the loop, which will cause some programs compiled by
GCC
to be less performance-optimal than before. 
%However, GCC developers feel
%that it is worthwhile considering the slow-down caused by the original loop.

\subsection{Accuracy Results}
\label{sec:result_acc}

%\input section/tab_top5
{\color{red}
To answer \textbf{RQ2}, we apply
statistical performance debugging \cite{SongOOPSLA2014} to all benchmarks
and then apply \Tool to the top 5 ranked loops\footnote{Some extracted benchmarks
have fewer than 5 loops. We simply apply \Tool to all loops in these cases.}.
}
{\color{red}
Table \ref{tab:benchmarks} shows that \Tool is accurate, having five real
false positives and 17 benign false positives in total for all the top five loops
of the \allbugs benchmarks.

%TODO for Linhai: 
%1. check the accuracy of the next two paragraphs
%2. fill in the XXX and XXX in the next paragraph
%3. change the false positive numbers in Table
%4. change the true false positive and benign false positive numbers in other part of the paper

The five real false positives happen for benchmarks GCC\#8805 and GCC\#12322.
These five loops are identified as
redundant and resultless by \Tool. However, they
are very difficult to fix and hence are not patched.
Since these five loops are ranked higher than the true positive loops by
statistical debugging, they could cost (waste) \Tool users some effort during 
performance diagnosis. Future work could try to prune these false positives
by estimating the fix difficulty.

\Tool also reports 17 benign false positives. These are 
correctly identified inefficient loops.
However, they are not the main contributors to the perceived
performance problems, and hence are not
fixed in patches. 
Different from the five false positives discussed above, these 17 loops are
ranked {\bf lower} than the true positive loops by statistical debugging, as
well as by profillers,
because of their low contributions to the perceived performance problems.
Consequently, they will not cause extra diagnosis effort for \Tool users.
}

The accuracy of \Tool benefits from its run-time analysis.
%{\color{red} For example, there are 26 
For example, our run-time analysis has correctly pruned out 24 false positives
in 0*1? inefficiency detection for our benchmarks. Each of these 24 loops is a
top-5 suspicious loop in one of our benchmarks; it only generates side effects
in its last iteration, and hence is identified as 0*1? by static analysis. 
Without run-time information, \Tool would judge
all of them as inefficient (0*1? resultless). Fortunately,
\Tool run-time counts the number of iterations of each loop instance and
correctly identifies them as false positives. Similarly, \Tool run-time analysis
helps prune out 19 false positives for [0$|$1]* loop identification.

\Tool can also help improve the accuracy of statistical debugging in
identifying which loop is the root-cause loop.
For example, the real root-cause loop of Apache\#34464 and GCC\#46401 both
rank number two by the statistical performance diagnosis tool.
Fortunately,
\Tool can tell that the number one loops in both cases do not contain
any form of inefficiency, resultless or redundancy. 

\subsection{Performance}
\label{sec:result_perf}

\input section/overhead_tab.tex

{\color{red}
To answer \textbf{RQ3}, we evaluate the run-time performance of
applying \Tool to the real root-cause loop. 
}
As shown in Table \ref{tab:performance}, 
the performance of \Tool is good. The overhead is consistently under or around 5\% 
except for one benchmark, Mozilla\#347306. 
We believe \Tool is promising for potential production
run usage.
We can easily further lower the overhead through sparser sampling.
%the current diagnosis results are obtained by running the
%program only \textbf{once} under the problem-triggering workload.
%If we use sparser sampling, more failure runs will be needed to obtain good
%diagnosis results.

As we can also see from the table, our performance optimization discussed in 
Section \ref{sec:inst} and \ref{sec:perf} has helped.

The performance benefit of sampling is huge.
Without sampling, redundancy
analysis lead to over 100X slowdown for six benchmarks.
The benefit of static optimization is also non-trivial. 
For example, for MySQL\#15811 and MySQL\#27287, static analysis alone can
judge that they do not contain cross-iteration redundancy (i.e., no run-time 
overhead): the computation of 
each iteration depends on loop induction variables, which are naturally different
in different iterations. 
As another example, the buggy loops of MySQL\#27287 and MySQL\#15811 access 
arrays. 
After changing to tracking the initial and ending memory-access addresses
of the array, instead of the content of the whole array accesses,
the overhead is reduced from 12.18\% to 0.20\% for MySQL\#27287, 
and from 20.53\% to 0.35\% for MySQL\#15811 respectively. 
%(sampling is conducted consistently here). 

\subsection{Parameter Setting and Sensitivity}
\label{sec:sensi}
\noindent\textbf{Sampling rates}
We have tried different sampling rates for redundancy analysis.
Intuitively, sparser sampling leads to lower overhead but worse diagnosis
results. Due to space constraints, we briefly summarize the results below.

When we lower the sampling rate from 1/100 to 1/1000 
in cross-loop redundancy analysis,
among all the benchmarks in Table \ref{tab:performance},
Mozilla\#416628 incurs the largest overhead (merely 4.51\%). 
When we lower the sampling rate from 1/1000 to 1/10000
in cross-iteration redundancy analysis,
among all the benchmarks in Table \ref{tab:performance},
Mozilla\#347306 has the largest overhead (merely 7.38\%). 
%In fact, Mozilla347306 is the only one whose overhead is larger than 2\%. 
In both cases,
the diagnosis results remain 
the same for all but GCC\#12322, where too few samples are available
to judge redundancy.
%If we increase the sampling rate to 1/100, we will 
%have two benchmarks whose overhead is larger than 30\%.

\noindent\textbf{Resultful and redundancy rate}
Our default setting should work in most cases.
In fact, the diagnosis results are largely {\bf in}sensitive to the threshold
setting. For example, the results would remain the same when
changing the redundancy rate threshold from 0.5 to any value between about
0.1 and 0.66. We will have 1 more false negative and 1 fewer true false positive, 
when the rate is 0.7. The trend is similar for resultless loop checking. 
Developers can adjust these thresholds. 
They can even get rid of thresholds, and only
use the raw values of resultful/redundancy rates to understand
the absolute and relative (in)efficiency nature of suspicious 
loops. Based on our experiments, the difference between efficient and inefficient
loops is obvious based on these rates.


