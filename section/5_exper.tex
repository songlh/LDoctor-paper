\section{Evaluation}
\label{sec:experiment}

\subsection{Methodology}
\label{sec:result_meth}
%Please discuss the potential usage scenarios of \Tool
%How you will use it together with other tools 


\begin{table}
  \centering
  \small
  \newcommand{\Yes}[1]{\checkmark{}$_#1$}
  \newcommand{\No}[0]{-}
  \begin{tabular}{lcccc}
    \toprule
                         &      	   &                        & {\bf Root}   &          \\
   {\bf BugID}           &  {\bf KLOC}     &  {\bf P. L.}           & {\bf Cause}  & {\bf Fix}\\
   \midrule
   Mozilla347306         & 88              & C                      &  0*1?        & C     \\
   Mozilla416628         & 105             & C                      &  0*1?        & C     \\
   Mozilla490742         & N/A             & JS                     &  C-I         & B       \\
   Mozilla35294          & N/A             & C++                    &  C-L         & B        \\ 
   Mozilla477564         & N/A             & JS                     &  C-L         & M       \\
   \midrule 
   MySQL27287            & 995             & C++                    &  0*1?,C-L        & C     \\
   MySQL15811            & 1127            & C++                    &  C-L         & M \\ 
   \midrule    
   Apache32546           & N/A             & Java                   &  C-I         & B  \\
   Apache37184           & N/A             & Java                   &  C-I         & M  \\
   Apache29742           & N/A             & Java                   &  C-L         & B \\ 
   Apache34464           & N/A             & Java                   &  C-L         & M  \\
   Apache47223           & N/A             & Java                   &  C-L         & B \\
   \midrule
   GCC46401              & 5521            & C                      &  [0$|$1]*    & S   \\
   GCC1687               & 2099            & C                      &  C-I         & M \\
   GCC27733              & 3217            & C                      &  C-I         & M \\
   GCC8805               & 2538            & C                      &  C-L         & B\\
   GCC21430              & 3844            & C                      &  C-L         & M \\
   GCC12322              & 2341            & C                      &  1*          & S\\
\bottomrule
   \end{tabular}
  %\nocaptionrule
  \caption{Benchmark information.
  (N/A: we skip the size of benchmarks that are extracted from real-world applications.
  Root cause ``C-I'' is short for cross-iteration redundancy.
  Root cause ``C-L'' is short for cross-loop redundancy.
  C, B, M, and S represent different fix strategies, as discussed in
  Table \ref{tab:root}.
  ) 
 }
  \label{tab:benchmarks}
\end{table}

\paragraph{Implementation and Platform}
We implement \Tool in LLVM-3.4.2 \cite{llvm}, and conduct our
%Our implementation consists of 17654 lines of C++ code, 
%with 4748 for instrumenter, 674 for resultless analysis, 
%5802 for redundancy analysis, 
%and the remaining 6430 for common utilities. 
experiments on a i7-960 machine, with Linux 3.11 kernel. 

\paragraph{Benchmarks}
We use 18 out of the 45 bugs listed in Table \ref{tab:root} as our 
evaluation benchmarks. Among these 18, seven are extracted from Java or JavaScript
programs and re-implemented in C++, as \Tool currently only handles C/C++
programs; one is extracted from a very old version of Mozilla.
%These 18 bugs include all the benchmarks used in the recent
%statistical performance debugging work~\cite{SongOOPSLA2014}.
The remaining bugs listed in Table \ref{tab:root} are much more difficult to
use as benchmarks, 
either because they depend on special hardware/software environment
or because they involve too complex data structures to extract. 
Overall, these 18 bugs cover a wide variety of performance root causes, as 
shown in Table~\ref{tab:benchmarks}. 

\paragraph{Metrics}
Our experiments are designed to evaluate \Tool from three main aspects:
\begin{itemize}
\item Coverage. Given our benchmark suite that covers a wide variety
of real-world root causes, can \Tool identify all those root causes?
\vspace{-0.05in}

\item Accuracy. 
When analyzing non-buggy loops, will \Tool generate any false positives?
\vspace{-0.05in}

\item Performance. 
What is the run-time overhead of \Tool?
\end{itemize}

\paragraph{Evaluation settings}
The imagined usage scenario of \Tool is that one will apply \Tool to identify
detailed root causes and provide fix-strategy suggestion for a small number of
suspicious loops that are most correlated with the specific performance problem.

Our evaluation uses existing statistical performance diagnosis
tool \cite{SongOOPSLA2014} to process a performance problem and identify 
one or a few suspicious loops for \Tool to analyze.
For 14 out of the 18 benchmarks, statistical performance debugging identifies the
real root-cause loop as the top ranked suspicious loop. For the remaining
benchmarks, the real root-cause loops are ranked number 2, 2, 4, and 10.
%Overall, we believe future tools can accurately identify the most one or a couple
%of suspicious loops.

\begin{table}
  \centering
  \small
  \newcommand{\Yes}[0]{\checkmark}
  \newcommand{\No}[0]{-}
  \begin{tabular}{lcc}
    \toprule
                            	&  {\bf Reported}             &{\bf Fix }                     \\
   {\bf BugID}                  &  {\bf Root Cause}           &{\bf Suggestion}             \\
   \midrule
   Mozilla347306                & \Yes                        & \Yes                                          \\
   Mozilla416628                & \Yes                        & \Yes                                         \\
   Mozilla490742$_a$            & \Yes                        & \Yes                                             \\
   Mozilla35294$_a$             & \Yes                          & \Yes                                              \\ 
   Mozilla477564$_a$            & \Yes                          & \Yes                                            \\
   \midrule 
   MySQL27287                   & \Yes                          & \ding{55}                                         \\
   MySQL15811                   & \Yes                          & \Yes                                      \\ 
   \midrule    
   Apache32546                  & \Yes                          & \Yes                                      \\
   Apache37184$_a$              & \Yes                          & \Yes                                        \\
   Apache29742$_a$              & \Yes                          & \Yes                                       \\ 
   Apache34464                  & \Yes                          & \Yes                                        \\
   Apache47223                  & \Yes                          & \Yes                                       \\
   \midrule
   GCC46401                     & \Yes                     & \Yes                                      \\
   GCC1687                      & \Yes                          & \Yes                                     \\
   GCC27733$_a$                 & \Yes                          & \Yes                                     \\
   GCC8805                      & \Yes                          & \Yes                                  \\
   GCC21430                     & \Yes                          & \Yes                                     \\
   GCC12322                     & \Yes                         & \ding{55}                                   \\
\bottomrule
   \end{tabular}
  %\nocaptionrule
  \caption{Coverage Results.}
  \label{tab:cover}
\end{table}


To evaluate the coverage, accuracy, and performance of \Tool, we mainly conduct
three sets of evaluation. First, we apply \Tool to the real root-cause loop to
see if \Tool can correctly identify the root-cause category and provide
correct fix-strategy suggestion. Second, we apply
statistical performance debugging \cite{SongOOPSLA2014} to all our benchmarks
and apply \Tool to the top 5 ranked loops\footnote{Some extracted benchmarks
have fewer than 5 loops. We simply apply \Tool to all loops in these cases.}
to see how accurate \Tool is. Third, we evaluate the run-time performance of
applying \Tool to the real root-cause loop. 
 
For all benchmarks we use, real-world
users have provided at least one problem-triggering input in their on-line 
bug bugs. We use these inputs in our run-time analysis.

As discussed in Section \ref{sec:design}, our analysis contains 
several configurable thresholds. In our evaluation,
we use 0.001 as the \textit{resultless rate} threshold for identifying
0*1? loops, 0.01 as the \textit{resultless rate} threshold for identifying 
[0$|$1]* loops, 0.5 as the \textit{cross-loop redundancy rate}, and 
2 as the \textit{cross-iteration redundancy rate} (i.e., 
the number of distinct iterations is less than half of the total iterations).

All the analysis and performance results presented below regarding
cross-loop analysis is obtained using $1/100$ sampling rate; all the
results regarding cross-iteration analysis is obtained using $1/1000$ sampling
rate. We use sparser sampling rate in the latter case, because there tend to
be more loop iterations than loop instances.
All our diagnosis results require only \textbf{one} run under the 
problem-triggering input.

\subsection{Coverage Results}
\label{sec:coverage}
Overall, \Tool provides good diagnosis coverage, as shown in Table~\ref{tab:cover}. 
\Tool identifies the correct root cause for \textbf{all} 18 benchmarks, and 
suggests fix strategies that exactly match what developers took in practice
for 16 out of 18 cases. There are only two cases where \Tool fails to suggest
the fix strategy that developers used. For MySQL27287, the root-cause loop
is both cross-loop redundant and 0*1? inefficient. \Tool suggests both changing
data structures and memoization as fix strategies. In practice, the developers
find a new data structure that can eliminate both root causes.
For GCC12322, \Tool correctly tells that the loop under study
does not contain any form of inefficiency and produce results in every 
iteration, and hence fails to suggest any fix strategy. In practice, GCC
developers decide to skip the loop, which will cause some programs compiled by
GCC
to be less performance-optimal than before. However, GCC developers feel
that it is worthwhile considering the performance impact of the original loop
to the GCC compilation process.
Providing this type of
fix strategy suggestion goes beyond the capability of \Tool.

\subsection{Accuracy Results}
\label{sec:result_acc}

\begin{table}
  \centering
  \small
  \newcommand{\Yes}[0]{\checkmark}
  \newcommand{\No}[0]{-}
  \begin{tabular}{lcccccc}
    \toprule
   {\bf BugID}           &  {\bf 0*1?}          &  {\bf [0$|$1]*}             &{\bf C-I$_b$}            &{\bf C-I$_m$}          &   {\bf C-L}     & {\bf Total}  \\
   \midrule
   Mozilla347306         &   -                  & -                           & -                       & -                     &   -             & -\\
   Mozilla416628         &   -                  & -                           & -                       & -                     &   -             & -\\
   Mozilla490742$_a$     &   -                  & -                           & -                       & -                     &   -             & -\\
   Mozilla35294$_a$      &   -                  & -                           & -                       & -                     &   -             & -\\ 
   Mozilla477564$_a$     &   -                  & -                           & -                       & -                     &   -             & -\\
   \midrule 
   MySQL27287            &   -                  & 0$_1$                       & -                       & -                     &   -             & 0$_1$\\
   MySQL15811            &   -                  & -                           & -                       & -                     &   -             & -\\ 
   \midrule    
   Apache32546           &   -                  & -                           & -                       & -                     &   -             & -\\
   Apache37184$_a$       &   -                  & -                           & -                       & -                     &   -             & -\\
   Apache29742$_a$       &   -                  & -                           & -                       & -                     &   -             & -\\ 
   Apache34464           &   -                  & -                           & -                       & -                     &   -             & -\\
   Apache47223           &   -                  & -                           & -                       & -                     &   -             & -\\
   \midrule
   GCC46401              &   -                  & 0$_1$                       & -                       & -                     &   -             & 0$_1$\\
   GCC1687               &   -                  & -                           & -                       & -                     &   -             & -\\
   GCC27733$_a$          &   -                  & -                           & -                       & -                     &   -             & - \\
   GCC8805               &   -                  & 0$_2$                       & -                       & -                     &   -             & 0$_2$\\
   GCC21430              &   0$_1$              & 0$_3$                       & -                       & 0$_1$                 &   0$_1$         & 0$_6$\\
   GCC12322              &   0$_1$              & 0$_1$                       & -                       & 0$_1$                 &   0$_1$         & 0$_4$\\
\bottomrule
   \end{tabular}
  %\nocaptionrule
  \caption{False positives of \Tool, when applying to top 5 loops reported by 
    statistical performance diagnosis for each benchmark. `-' represents zero false positive.
    Other cells report real false positives and benign false positives, which is in the
    subscript.
}
  \label{tab:top5}
\end{table}

As shown in Table \ref{tab:top5}, \Tool is accurate, having 0 real
false positive and 14 benign false positives for all the top 5 loops.

Here, benign false positives mean that the \Tool analysis result is true ---
some loops are indeed cross-iteration/loop redundant or indeed producing
results in only a small portion of all the iterations. However, those
problems are \textit{not} fixed by developers in their performance patches. 

There are several reasons for these benign performance problems. 
The main reason is that they are not the main contributor to the 
performance problem perceived by the users. This happens to 12 out of the
14 benign cases. In fact, this is not really a problem for \Tool in 
real usage scenarios, because statistical debugging can accurately
tell that these loops are not top contributors to the performance
problems.
The remaining two cases happen when fixing the 
identified redundant/resultless problems
are very difficult and hence developers decide not to fix them.

The accuracy of \Tool benefits from its run-time analysis.
For example, 4 benchmarks contain loops that only generate side-effect
in the last iteration among their top 5 suspicious loops. However,
these loops are actually not inefficient because the portion of
resultless iterations is small. 
%TODO a table 6 was deleted from oopsla submission

The good accuracy of \Tool can actually help improving the accuracy of
identifying which loop is the root cause loop.
For example, the real root-cause loop of Apache34464 and GCC 46401 both
rank number two by the statistical performance diagnosis tool.
\Tool can tell that the number one loops in both cases do not contain
any form of inefficiency, resultless or redundancy. This result can
potentially used to improve the accuracy of identifying root cause loops.


\subsection{Performance}
\label{sec:result_perf}

\begin{table}
  \centering
  \scriptsize
  \newcommand{\Yes}[1]{\checkmark{}$_#1$}
  \newcommand{\No}[0]{-}
  \begin{tabular}{lccccc}
    \toprule
	    & \multicolumn{3}{c}{\Tool} & \multicolumn{2}{c}{w/o optimization} \\
     \cmidrule(lr){2-4}
     \cmidrule(lr){5-6}
     {\bf BugID}  & {\bf Resultless}  &  {\bf C-L R. } & {\bf C-I R. }  & {\bf C-L R.}  & {\bf C-I R. } \\
    \midrule
    Mozilla347306 &  1.07\%           &  22.40\%       &  10.17\%       & 304.37{\bf X} & 468.74{\bf X} \\ 
    Mozilla416628 &  0.80\%           &  4.10\%        &  2.99\%        & 567.51{\bf X} & 85.6{\bf X} \\
    \midrule
     MySQL27287   & $\sim$0           &   1.66\%       &   -            & 109.55{\bf X} & 352.07{\bf X} \\
     MySQL15811   &  -                &   0.03\%       &   -            & 227.04{\bf X} & 424.44{\bf X} \\
    \midrule
      GCC46401    & 3.12\%         & 3.80\%            &  5.95\%        & 21.07{\bf X}  & 38.44{\bf X}\\ 
      GCC1687     & -              & /                 &  $\sim$0       &   /           & 142.29{\bf X} \\
      GCC27733$_a$    & $\sim$0        & /                 &  4.73\%        &   /           & 17.41{\bf X}     \\
      GCC8805     & -              & $\sim$0           & $\sim$0        & 2.22{\bf X}   &  3.52{\bf X}\\
      GCC21430    & -              & 5.46\%            &   0.69\%       & 107.20{\bf X} & 159.89{\bf X} \\
      GCC12322    & -              & 1.75\%            &  $\sim$0       & 21.07{\bf X}  & 38.44{\bf X} \\
   \bottomrule
   \end{tabular}
  %\nocaptionrule
  \caption{Run-time overhead of applying \Tool to the buggy loop, with and
    without optimizations. 
    Only results from non-extracted benchmarks are shown. 
  -: static analysis can figure out the results and hence no dynamic analysis is conducted.
  /: not applicable. }
  \label{tab:performance}
\end{table}


\comment{
\begin{table}
  \centering
  \scriptsize
  \newcommand{\Yes}[1]{\checkmark{}$_#1$}
  \newcommand{\No}[0]{-}
  \begin{tabular}{lccccc}
    \toprule
                       & {\bf Resultless} &  \multicolumn{2}{c}{\bf C-L R. } & \multicolumn{2}{c}{\bf C-I R.}\\
      \cmidrule(lr){3-4}    
      \cmidrule(lr){5-6}   
     {\bf BugID}       &                & Not               &  S.+O.            & Not                & S.+O.  \\
     \midrule
     Mozilla347306     & 1.07\%         & 303.37{\bf X}     &  22.40\%       & 467.74{\bf X}         & 10.17\%      \\
     Mozilla416628     & 0.80\%         & 56.51{\bf X}      &  4.10\%        & 84.6{\bf X}           & 2.99\%\\
     \midrule
     MySQL27287        & $\sim$0        & 108.55{\bf X}     &  1.66\%        & 351.07{\bf X}         & - \\
     MySQL15811        &  -             & 226.04{\bf X}     &  0.03\%        & 423.44{\bf X}         & - \\
     \midrule
      GCC46401         & 3.12\%         & 20.07{\bf X}      &  3.80\%        & 37.44{\bf X}          & 5.95\%\\ 
      GCC1687          & -              & /                 &  /             & 141.29{\bf X}         & $\sim$0 \\
      GCC27733         & $\sim$0        & /                 &  /             & 16.41{\bf X}          & 4.73\%         \\
      GCC8805          & -              & 1.22{\bf X}       & $\sim$0        & 2.52{\bf X}           & $\sim$0\\
      GCC21430         & -              & 106.20{\bf X}     & 5.46\%         & 158.89{\bf X}         & 0.69\%\\
      GCC12322         & -              & 20.07{\bf X}      & 1.75\%         & 37.44{\bf X}          & $\sim$0\\
  \bottomrule
   \end{tabular}
  %\nocaptionrule
  \caption{Run-time overhead of applying \Tool to the buggy loop. 
    Only performance results from non-extracted benchmarks are shown here. 
  Not: not applying the performance optimization in \Tool.
  -: static analysis can figure out the results and hence no dynamic analysis is conducted.
  /: not applicable. }
  \label{tab:performance}
\end{table}
}

As shown in Table \ref{tab:performance}, 
the performance of \Tool is good. The overhead is consistently under or around 5\% 
except for one benchmark, Mozilla347306. We believe \Tool is promising for potential production
run usage. Of course, if we apply \Tool to multiple loops simultaneously,
the overhead will be higher. However, the current results are obtained by running the
program only \textbf{once} under the problem-triggering workload. The sampling nature
of \Tool will allow us to keep the overhead low at the exchange of running the program for
a couple of more times, if needed.

As we can also see from the table, our performance optimization discussed in 
Section \ref{sec:inst} and \ref{sec:perf} has contributed
a lot to the good performance of \Tool.

Without sampling, while still applying our static optimization, our redundancy
analysis would lead to over 100X slowdown for five benchmarks.

The buggy loops of MySQL\#27287 and MySQL\#15811 access arrays. 
After changing to tracking the initial and ending memory-access addresses
of the array, instead of the content of the whole array accesses,
the overhead is reduced from 11.77\% to 1.66\% for MySQL\#27287, 
and from 20.46\% to 0.03\% for MySQL\#15811 respectively 
(sampling is conducted consistently here). 

The side-effect of the buggy loop for MySQL15811 is to calculate the length of a string. 
The variable representing the length is an induction variable. 
The side-effect of the buggy loop for MySQL27287 is to calculate the index of a searched target, 
and the variable representing the index is also an induction variable. 
We can reply on static analysis to figure out that these two loops are not cross-iteration redundant.

We also tried sampling with different sampling rates.
For cross-loop redundancy, we also conduct experiments under sampling rate 1
out of 1000. 
Both run-time overhead and collected sample will be reduced. Mozilla347306 
is still the benchmark with largest overhead, but the overhead is reduced to 0.49\%. For two benchmarks, GCC8805 and GCC12322, we cannot sample 
more than 10 loop instances and hence cannot draw strong conclusion
about their root-cause type. For all other benchmarks, we can 
still have more than 10 loop instances and get exactly the same 
root-cause analysis results presented above.

We also conduct cross-iteration redundant experiment under different sampling rates. When the sampling rate is 1 out of 100, there are two benchmarks, whose runtime overhead is larger than 30\%. The overhead for Mozilla347306 is 59.57\%, 
and the overhead of GCC21430 is 30.65\%. When changing sampling rate to 1
out of 10000, Mozilla347306 has the largest overhead, which is 4.47\%. And mozilla347306 is the only one whose overhead is larger than 2\%. Except for 
GCC12322, all other benchmarks will have more than 100 iterations sampled, under the sample rate is 1 out of 10000. Consequently, the same root-cause
analysis results will be reported for these benchmarks as the ones presented
above.
