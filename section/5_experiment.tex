\section{Experiment}
\label{sec:experiment}

\subsection{Methodology}


\begin{table}
  \centering
  \small
  \newcommand{\Yes}[1]{\checkmark{}$_#1$}
  \newcommand{\No}[0]{-}
  \begin{tabular}{lccccl}
    \toprule
                    &      	   &                        &{\bf Loop }  &  {\bf  \# of}   & {\bf Root}  \\
   {\bf BugID}      &  {\bf KLOC}  &  {\bf P. L.}           &{\bf Ranks}  &  {\bf  Loops}   & {\bf Cause} \\
   \midrule
   Mozilla347306    & 88           & C                      & 1               & 5               &  W. 0*1?     \\
   Mozilla416628    & 105          & C                      & 1               & 5               &  W. 0*1?     \\
   Mozilla490742    & N/A          & JS                     & 1               & 1               &  C-I. R.       \\
   Mozilla35294     & N/A          & C++                    & 1               & 2               &  C-L. R.        \\ 
   Mozilla477564    & N/A          & JS                     & 1               & 2               &  C-L. R.       \\
   \midrule 
   MySQL27287       & 995          & C++                    & 1               & 5               &  W. 0*1?     \\
   MySQL15811       & 1127         & C++                    & 1               & 5               &  W. 1* \\ 
   \midrule    
   Apache32546      & N/A          & Java                   & 1               & 1               &  C-I. R.  \\
   Apache37184      & N/A          & Java                   & 1               & 1               &  C-I. R.  \\
   Apache29742      & N/A          & Java                   & 1               & 2               &  C-L. R. \\ 
   Apache34464      & N/A          & Java                   & 2               & 3               &  C-L. R.  \\
   Apache47223      & N/A          & Java                   & 1               & 2               &  C-L. R. \\
   \midrule
   GCC46401         & 5521         & C                      & 2               &   5             &  W. [0$|$1]*   \\
   GCC1687          & 2099         & C                      & 1               &   5             &  C-I. R. \\
   GCC27733         & 3217         & C                      & 1               &   5             &  C-I. R. \\
   GCC8805          & 2538         & C                      & 4               &   5             &  C-L. R.\\
   GCC21430         & 3844         & C                      & 1               &   5             &  C-L. R. \\
   GCC12322         & 2341         & C                      & $>5$            &   5             &  Other\\
\bottomrule
   \end{tabular}
  %\nocaptionrule
  \caption{Benchmark Information. (Loop Ranks: buggy loop ranking in results from the statistical model.
  \# of Loops: the number of loops we analyze for each benchmark. 
  N/A: we re-implement benchmarks not in C/C++, or depending on very old libraries and operating systems. 
  W. is short for Resultless, R. is short for Redundancy, C-I. is short for cross-iteration, and C-L. is short for cross-loop. )}
  \label{tab:benchmarks}
\end{table}

{\bf Implementation and Platform}
We implement our analysis in LLVM-3.4.2. 
Our analysis consists of 17654 lines of C++ code, 
with 4748 for instrumenter, 674 for resultless analysis, 
5802 for redundancy analysis, 
and the remaining 6430 for common utilities. 
All our experiments are conducted on a i7-960 machine, with Linux 3.11 kernel. 

{\bf Benchmark Selection}
Beside all loop-related performance bugs used in~\cite{SongOOPSLA2014}, 
we install one new bug GCC27733, and re-implement 5 bugs. 
We fail to install or extract other bugs, 
either because they depend on some special platforms or very old libraries, 
or because they involve very complex data structures. 
In total, we use 18 benchmarks in our experiments. 
For all real bugs, we follow their default building setting. For all re-implemented bugs,
we build them by using ``-O0 -g''. 
Detailed benchmark information is shown in Table~\ref{tab:benchmarks}. 
For all bugs used in~\cite{SongOOPSLA2014} and the new bug we install (GCC27733), 
we follow the same methodology discussed in~\cite{SongOOPSLA2014} to run the statistical model. 
We run our analysis on top 5 loops, 
when there are more than 5 loops reported by the statistical model, 
or all loops, when there are less than 5 loops reported. 
For bugs we re-implement, we run our analysis on all executed loops, 
except loops bounded by constant iteration numbers.
We rank all these loops by their total iteration number. 
The total number of loop we analyze is 64.   

{\bf Experiment Input} 
For all benchmarks we use, 
users provided at least one bad input when reporting the bugs. 
When collecting dynamic information, 
we run instrumented programs by using the bad inputs provided by users.  

\subsection{Evaluation Results}

\begin{table*}
  \centering
  \tiny
  \newcommand{\Yes}[1]{\checkmark{}$_#1$}
  \newcommand{\No}[0]{-}
  \begin{tabular}{lccccccccccccccc}
    \toprule
              &  \multicolumn{5}{c}{\bf Static Resultless Type} & \multicolumn{5}{c}{\bf Dynamic Information} & \multicolumn{5}{c}{\bf Overhead}\\
\cmidrule(lr){2-6} \cmidrule(lr){7-11} \cmidrule(lr){12-16}
   {\bf BugID}      & Loop0 & Loop1 &  Loop2 &   Loop3 &   Loop4 & Loop0     &   Loop1 &   Loop2 &  Loop3 &   Loop4 &   Loop0 & Loop1 &   Loop2 &  Loop3 &  Loop4 \\
    \midrule
    Mozilla347306   & 0*1?          & \No   & \No    &   \No          & \No     & $2.00*10^4$ & /       & /       &  /     &  /      &   1.07\%   &   /   &   /     &  /      &  /\\
    Mozilla416628   & 0*1?          & \No   & \No    & \No            &   \No   & $1.82*10^3$ & /       & /       &  /     &  /      &   0.65\%   &   /   &   /     &  /       &  /\\
    Mozilla490742   & \No           & /     & /      & /              &   /     & /           & /       & /       &  /     &  /      &   /        &   /   &   /     &  /       &  /\\
    Mozilla35294    & \No           & \No   & /      & /              &   /     & /           & /       & /       &  /     &  /      &   /        &   /   &   /     &  /       &  /\\
    Mozilla477564   & [0$|$1]*(\No) & \No   & /      & /              &   /     & 0.99        & /       & /       &  /     &  /      &   1.32\%   &   /   &   /     &  /       &  /\\
    \midrule
    MySQL27287      & 0*1?          & \No   & \No    & [0$|$1]*(\No)   &   \No  & $6.55*10^4$ & /       & /        &  0(\No)  &  /         &  $\sim$0    &   /      &   /  &  $\sim$0 &  /\\
    MySQL15811      & \No           & \No   & \No    & \No      &   \No   & /        & /      & /       &  /       &  /       &   /      &   /      &   /      &  /       &  /\\
    \midrule
    Apache32546     & \No      & /        & /        & /        &   /     & /        & /        & /           &  /       &  /       &   /      &   /      &   /      &  /       &  /\\
    Apache37184     & \No      & /        & /        & /        &   /     & /        & /        & /           &  /       &  /       &   /      &   /      &   /      &  /       &  /\\
    Apache29743     & \No      & \No      & /        & /        &   /     & /        & /        & /           &  /       &  /       &   /      &   /      &   /      &  /       &  /\\
    Apache34464     & \No      & 0*1?(\No)& \No      & /        &   /     & /        & 51.05    & /           &  /       &  /       &   /      &   $\sim$0&   /      &  /       &  /\\
    Apache47223     & \No      & \No      & /        & /        &   /     & /        & /        & /           &  /       &  /       &   /      &   /      &   /      &  /       &  /\\
    \midrule
    GCC46401        & \No      & [0$|$1]*   & [0$|$1]*(\No)   & [0$|$1]*(\No)   & [0$|$1]*(\No)     & /        & 0     & 0.99     &  0(\No)       &  0.99 &   /    &  $\sim$0  & $\sim$0   & $\sim$0  & $\sim$0  \\ 
    GCC1687         & \No      & \No      &   0*1?(\No)     & \No      &   \No   & /        & /        & 41.84    &  /       &  /           &   /      &   /      & $\sim$0  &  /       &  /\\
    GCC27733        & 0*1?(\No)& \No      &   0*1?(\No)     & \No      &   \No   & 35.60    & /        & 27.62    &  /       &  /           & $\sim$0         &   /      &  $\sim$0         &  /     &  /\\
    GCC8805         & 0*1?(\No)& [0$|$1]*(\No)& [0$|$1]*(\No)& \No     &   \No   & $3.29*10^2$         & 0(\No)   & 0(\No)  &  /            & /       & 0.91\% &   0.02\% &  $\sim$0 &  /  &   /\\
    GCC21430        & \No      & [0$|$1]*(\No)& [0$|$1]*(\No)& 0*1?(\No)& [0$|$1]*(\No) & /            & 0(\No)   & 0(\No)  & $5.00*10^3$(\No)     &  0    &   /      &   $\sim$0 &   $\sim$0 &   $\sim$0 &  $\sim$0 \\
    GCC12322        & [0$|$1]*(\No)& 0*1?(\No)& 0*1?(\No)   & [0$|$1]*(\No) &   \No   & 0  & 157.74 & $3.31*10^3$(\No)   &  0.22     &  /     &  5.98\%        & $\sim$0     & $\sim$0      &  2.01\%   &  /\\
  \bottomrule
   \end{tabular}
  %\nocaptionrule
  \caption{Experimental Results for Resultless Analysis. 
  ( (\No) means false positives. 
   In the dynamic part, we report average iteration number for 
  0*1?, and ratio of working iteration number for [0$|$1]*.)}
  \label{tab:workless}
\end{table*}

In this section, we run our analysis on all 64 loops. 
We will discuss the effectiveness and performance for each of our techniques. 

\underline{\textit{Resultless Analysis}}
The results for resultless analysis is shown in Table~\ref{tab:workless}.
3 benchmarks are caused by 0*1?, and 1 benchmark is caused by [0$|$1]*. 
They are all identified by related static analysis. Our static analysis does not have false negatives. 
Besides these 4 loops, there are other 20 loops identified as either 0*1? or [0$|$1]*. 
We manually check these 20 loops, and we find that our static analysis does not make any mistakes. 
Since these 20 loops are not related to root causes, we still consider them as false positives. 
Without dynamic information, static resultless analysis will have a large false positive number. 

We consider average iteration number larger than 1000 is 0*1?, and ratio of iterations generating results less than 0.01 is [0$|$1]*. 
There are 8 false positives after we combine static analysis with dynamic information. 
One more heuristic we can apply is when a loop is identified as resultless loop, 
we will not consider resultless loops with lower ranks. After this, we will have 3 false positives, 
which are the second loop for GCC\#8805 and GCC\#21430, and the third loop for GCC\#12322. We will not have any false negatives. 

Collecting dynamic resultless information does not incur a large overhead. The largest overhead is 5.98\%, and for all other cases, the overhead is less than 3\%. 


\begin{table}
  \centering
  \small
  \newcommand{\Yes}[1]{\checkmark{}$_#1$}
  \newcommand{\No}[0]{-}
  \begin{tabular}{lccccc}
    \toprule
   {\bf BugID}        & {\bf L0}   &  {\bf L1}  & {\bf L2}            & {\bf L3} &  {\bf L4} \\
    \midrule
    Mozilla347306     & \No        &  \No       & \No                 & \No       & \No     \\
    Mozilla416628     & \No        &  \No       & \No                 & \No       & \No      \\
    Mozilla490742     & \checkmark &  /         & /                   & /         & /       \\
    Mozilla35294      & \No        &  \No       & /                   & /         & /      \\
    Mozilla477564     & \No        &  \No       & /                   & /         & /      \\
    \midrule
    MySQL27287       & \No         &  \No       &  \No                & \No      & \No       \\ 
    MySQL15811       & \No         &  \No       &  \No                & \No      & \No       \\
    \midrule
    Apache32546      & \checkmark  &  /         &  /                  & /        & /           \\
    Apache37184      & \No         &  /         &  /                  & /        & /           \\
    Apache29743      & \No        &  \No        &  /                  & /        & /           \\
    Apache34464      & \No        &  \No        &  \No                & /        & /           \\
    Apache47223      & \No        & \No         & /                   & /        & /           \\
    \midrule
    GCC46401          & \No        & \No        & \No                 & \No      & \No           \\
    GCC1687          & \No        & \No        & \No                 & \No      & \No        \\
    GCC27733         & \No        & \No        & \No                 & \No      & \No       \\
    GCC8805         & \No        & \No        & \No                 & \No      & \No        \\
    GCC21430         & \No        & \No        & \No                 & \No      & \No        \\
    GCC12322         & \No        & \No        & \No                 & \No      & \No         \\
    \bottomrule
   \end{tabular}
  %\nocaptionrule
  \caption{Experimental Results for System Call Checking. }
  \label{tab:sys}
\end{table}

\underline{\textit{Batchable System Call}}
The results for static batchable system call analysis is shown in Table~\ref{tab:sys}. 
Our static analysis does point out the two loops caused by batchable system calls, and it does not bring any false positives. 

\begin{table*}
  \centering
  \scriptsize
  \newcommand{\Yes}[1]{\checkmark{}$_#1$}
  \newcommand{\No}[0]{-}
  \begin{tabular}{lcccccccccc}
    \toprule
              &  \multicolumn{5}{c}{\bf Redundant-pair/Total-pair} & \multicolumn{5}{c}{\bf Overhead} \\
\cmidrule(lr){2-6} \cmidrule(lr){7-11} 
   {\bf BugID}      & Loop0 & Loop1 &  Loop2 &   Loop3 &   Loop4 & Loop0     &   Loop1   &   Loop2 &  Loop3 &   Loop4   \\
    \midrule
    Mozilla347306   & 0     & 0     &  0     &   0     &   0     &  22.4\%   &   55.24\% & 0.03\%  &$\sim$0 &  $\sim$0   \\
    Mozilla416628   & 0     & 0     &  0     &   0     &   0     &  2.74\%   &   0.07\%  & $\sim$0 &  0.03\%&  $\sim$0    \\
    Mozilla490742   & /     & /     & /      & /       &   /     & /         & /         & /       &  /     &  /      \\
    Mozilla35294    & 1     & /     & /      & /       &   /     & 19.32\%   & /   & /       &  /     &  /      \\
    Mozilla477564   & 1     & /     & /      & /       &   /     & 6.6\%     & /         & /       &  /     &  /    \\
    \midrule
    MySQL27287      & 1(\No) & 0     & 0      & 0      &   0     &           &     &        &   &        \\
    MySQL15811      & 0.98(\No) & 0  & 0      & 0      & 0       &           &         &           &       &         \\
    \midrule
    Apache32546     & /     & /     & /      & /       &   /     & /        & /        & /           &  /       &  /       \\
    Apache37184     & /     & /     & /      & /       &   /     & /        & /        & /           &  /       &  /       \\
    Apache29743     & 0.97  & /     & /      & /       &   /     & $\sim$0  & /        & /           &  /       &  /       \\
    Apache34464     & 0     & 1     & /      & /       &   /     & 23.81\%  & 21.35\%  & /           &  /       &  /      \\
    Apache47223     & 0.97  & /     & /      & /       &   /     & 21.05\%  & /        & /           &  /       &  /       \\
    \midrule
    GCC46401        & 0     & 0     & 0      & 0       &   0     & 34.43\%  & $\sim$0  &  0.14\%     & $\sim$0  &  $\sim$0  \\ 
    GCC1687         & /     & 0     & 0      & 0       &   0     & /        & $\sim$0  & $\sim$0     & 0.07\%   &  1.71\%   \\
    GCC27733        & /     & 0     & 0      & 0       &   0     & /        & $\sim$0  & $\sim$0     & $\sim$0  & $\sim$0    \\
    GCC8805         & 0     & 0     & 0      & 1       &   0     & 10.15\%  & $\sim$0  & $\sim$0     & $\sim$0  & $\sim$0        \\
    GCC21430        & 0.99  & /     & /      & 1(\No)       &   0     & 5.46\%   &  /       &  /          & $\sim$0  & $\sim$0        \\
    GCC12322        & 0     & 0     & 0      & /       &   1(\No)     & 2.57\%   & $\sim$0  & $\sim$0     &  /       & $\sim$0 \\
  \bottomrule
   \end{tabular}
  %\nocaptionrule
  \caption{Experimental Results for Cross-loop Redundancy Analysis.}
  \label{tab:CL-Results}
\end{table*}


\underline{\textit{Cross-loop Redundancy}}
As we discussed in Section~\ref{sec:cal}, we skip consecutive loop instance pairs, where two instances are both too short, 
or one instance is too short compared with the other. The concrete threshold we use in our experiment is that we skip pairs, 
where two instances are both less than 100 iterations, 
or the shorter one is less than 100 iterations and less than half of the longer instance's iterations. 

The result for cross-loop redundancy analysis under sampling rate 1/100 is shown in Table~\ref{tab:CL-Results}.  
We skip loops without outer loop, and loop recursively call the function containing itself.
All 7 buggy cross-loop redundant loops are identified. 
We have 4 false positives. 
We do not check whether the loop generates results or not, 
when we conduct cross-loop redundancy checking. We identify the first loop of MySQL\#27287 as cross-loop redundancy, which actually is a resultless loop.
The reason for the other 3 false positives is that there are repetitive data inside the input provided by the user. 
For example, the MYSQL\#15811 is triggered when loading a table from a file into database. The table contains 16384 records, and all of these records are identical string `1234567890'. 
Another example is GCC\#21430. The bad input provided by the user contain 4 nested macro definition, 
and after macro expansion, there will be 10000 identical lines of codes. 

There are two heuristics we could apply to cut false positives. Firstly, when a loop has already been identified as resultless, 
we will not consider it as redundancy. Secondly, when a loop with higher rank has been identified as redundancy, 
we will not consider loops with lower ranks as redundancy. 
After applying these two heuristics, we will have 2 false positives, which is the first loop for MySQL\#15811, and the 5th loop for GCC\#12322.

The overhead for cross-loop instrumentation is quite large. There are two loops, whose overhead is larger than 30\%.
 The reason is that each instance of these two loops only contain 1 or 2 iterations. 
 We may consider skip loops with small average iteration number before we conduct cross-loop instrumentation.  

\begin{table*}
  \centering
  \scriptsize
  \newcommand{\Yes}[1]{\checkmark{}$_#1$}
  \newcommand{\No}[0]{-}
  \begin{tabular}{lcccccccccc}
    \toprule
              &  \multicolumn{5}{c}{\bf Total-iteration/Distinct-iteration} & \multicolumn{5}{c}{\bf Overhead} \\
\cmidrule(lr){2-6} \cmidrule(lr){7-11} 
   {\bf BugID}      & Loop0 & Loop1 &  Loop2 &   Loop3 &   Loop4 & Loop0     &   Loop1 &   Loop2 &  Loop3 &   Loop4   \\
    \midrule
    Mozilla347306   &   1   &  1    &   1    &   1     &    1    &   10.17\% & -        & -      & $\sim$0& -    \\
    Mozilla416628   &   1   &  1    &   1    &   1     &    1    &   2.50\%  & 1.88\%   & 1.86\% & -        &  -     \\
    Mozilla490742   &   0   &  /    &   /    &   /     &    /    &   $\sim$0 & /        & /      &  /     &  /      \\
    Mozilla35294    &   1   &  1    & /      & /       &   /     &   -       & -        & /      &  /     &  /      \\
    Mozilla477564   &   1   &  1    & /      & /       &   /     &   $\sim$0 & 0.99\%   & /      &  /     &  /    \\
    \midrule
    MySQL27287      &   1   &  1    &  1     &   1     &   1     &    -      &          &        &  -      &        \\
    MySQL15811      &   1   &  1    &  1     &   1     &   1     &    -      &          &   -    &  -      &         \\
    \midrule
    Apache32546     &  47.45(\No) & /     & /      & /       &   /    & 3.85\%  & /        & /           &  /       &  /       \\
    Apache37184     &  3     & /     & /      & /       &   /         &  $\sim$0         & /        & /           &  /       &  /       \\
    Apache29743     &  1    & 1     & /      & /       &   /     &   0.14\%  &   $\sim$0       & /          &  /       &  /       \\
    Apache34464     &  1    & 1     & 1      & /       &   /     &    -      &          &            &  /       &  /      \\
    Apache47223     &  1    & 1     & /      & /       &   /     &    -      &  1.94\%   & /           &  /       &  /       \\
    \midrule
    GCC46401        &  1    & 1     & 1      & 1       &   1     &   38.30\%  & $\sim$0    & $\sim$0 & $\sim$0   & $\sim$0  \\ 
    GCC1687         & 1957.18 & 1   & 1      & 1       &   1     &  $\sim$0   & -         &  -       & -         & -         \\
    GCC27733        & 4.59    & 1   & 1      & 1       &   1     &  3.48\%    & $\sim$0   & 0.80\%   & -         & -         \\
    GCC8805         &  1    &  1    & 1      & 1       &   1     &  5.68\%    & 0.12\%    & $\sim$0  & $\sim$0   & -       \\
    GCC21430        &  1    & 1     & 191(\No)    & 1       &   1     &  0.69\%    & $\sim$0   & $\sim$0  & -         & -       \\
    GCC12322        & 1     & 1     &  1     & 3.30(\No) &   1     &  $\sim$0 & $\sim$0   & $\sim$0  & $\sim$0   & $\sim$0        \\
  \bottomrule
   \end{tabular}
  %\nocaptionrule
  \caption{Experimental Results for Cross-iteration Redundancy Analysis. (- in Overhead part means we get the result based on static analysis.) }
  \label{tab:CI-Results}
\end{table*}

The results for cross-iteration redundancy is shown in Table~\ref{tab:CI-Results}. 
The experiment is conducted under sampling rate 1/1000. 
The iteration number for the buggy loop of Mozilla\#490472 is only 50, and we do not have any sample for that bug.

We have 3 benchmarks caused by cross-iteration redundancy, and they are all identified by our analysis. 
There are 3 false positives reported by our analysis. We identify 
Apache\#32546 as cross-iteration redundancy, because we just monitored the parameter for IO functions, 
and we do not monitor the content of the buffer used by these IOs. 
The reason for the other 2 false positives is that there are repetitive data inside the input provided by the user. 

For many loops, side-effect instructions depend on induction variables, 
and we do not need dynamic analysis to know that there is no cross-iteration redundancy.   


\subsection{Analysis Combination}

\begin{table}
  \centering
  \small
  \newcommand{\Yes}[1]{\checkmark{}$_#1$}
  \newcommand{\No}[0]{-}
  \begin{tabular}{lcccc}
    \toprule
                    &{\bf Loop }   & {\bf Root}   & {\bf Diagnosis} & {\bf Diagnosis} \\
   {\bf BugID}      & {\bf Ranks}    & {\bf Cause}  & {\bf Rank }    & {\bf Result }   \\
   \midrule
   Mozilla347306    & 1            &  W. 0*1?     & 1             &  W. 0*1? \\
   Mozilla416628    & 1            &  W. 0*1?     & 1             &  W. 0*1?\\
   Mozilla490742    & 1            &  C-I. R.(B)  & 1             &  C-I. R.(B)\\
   Mozilla35294     & 1            &  C-L. R.     & 1             &  C-L. R.   \\ 
   Mozilla477564    & 1            &  C-L. R.     & 1             &  C-L. R.   \\
   \midrule 
   MySQL27287       & 1            &  W. 0*1?     & 1             &  W. 0*1? \\
   MySQL15811       & 1            &  W. 1*       & 1             &  C-L. R.\\ 
   \midrule    
   Apache32546      & 1            &  C-I. R.(B)  & 1             &  C-I. R.(B)\\
   Apache37184      & 1            &  C-I. R.     & 1             &  C-I. R.\\
   Apache29742      & 1            &  C-L. R.     & 1            &  C-L. R. \\ 
   Apache34464      & 2            &  C-L. R.     & 2            &  C-L. R.\\
   Apache47223      & 1            &  C-L. R.     & 1            &  C-L. R.\\
   \midrule
   GCC46401         & 2              &  W. [0$|$1]* & 2              &  W. [0$|$1]*  \\
   GCC1687          & 1              &  C-I. R.    & 1              &  C-I. R.\\
   GCC27733         & 1              &  C-I. R.    & 1              &  C-I. R.\\
   GCC8805          & 4              &  C-L. R.    & 4              &  C-L. R.\\
   GCC21430         & 1              &  C-L. R.    & 1              &  C-L. R.\\
   GCC12322         & $>5$           &  Other      & 4              &  C-I. R. \\
\bottomrule
   \end{tabular}
  %\nocaptionrule
  \caption{Combined results. Diagnosis Rank: the rank of the loop identified as root cause loop by our technique. 
  Diagnosis Result: the reason for performance loss our diagnosis tool report.}
  \label{tab:final}
\end{table}

In this section, we will discuss whether our techniques can enhance statistical fault localization for performance bugs. 

The first thing we need to do is to combine results from our different types of analysis. 
Our static analysis for batchable system call is very precise. 
If a loop is identified as batchable system call, 
we consider that loop is the buggy loop, and the root cause is batchable system call. 
Since we do not consider whether a loop generating results when we conduct redundant analysis, 
when a loop is identified as both redundancy and resultless, we will consider it as resultless. 
When a loop is identified by resultless, we will not consider resultless loops with lower rank. 
The same rule is also applied to redundancy. 
We think that resultless could be side-effect of redundancy, 
if there is a loop identifies as redundancy in top 5 loops, 
we will consider the root cause as redundancy.  

The combined result is shown in Table~\ref{tab:final}. 
We do not have correct diagnosis results for two bugs, 
MySQL\#15811 and GCC\#12322. The root cause for MySQL\#15811 is 1*, 
and we identify the buggy loop as cross-loop redundancy. The buggy loop for GCC\#12322 is not within top 5.  
Our tool report GCC\#12322 is caused by cross-iteration redundancy, and the buggy loop is the 4th loop. 

Compared with statistical debugging, our technique does not make results worse for all benchmarks. 
We improve the results for 3 benchmarks. We cut 1 false positives for Apache\#34464, 1 false positives for GCC\#46401, and 3 false positives for GCC\#8805.
Except MySQL\#15811, our technique can report the precise root cause when the buggy loop is ranked as 1st loop by the statistical model. 
