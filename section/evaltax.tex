\section{Evaluation of Taxonomy}
\label{sec:eval_taxonomy}

\input section/tab_appbug
\input section/tab_root

The root-cause taxonomy presented in Section \ref{sec:study} is the foundation
of \Tool. In this section,
we quantitatively 
assess the coverage, actionability, and generality of our taxonomy using
a set of
real-world inefficient loop problems collected by previous work
\cite{SongOOPSLA2014,PerfBug}.


\subsection{Methodology}

Previous work \cite{PerfBug,SongOOPSLA2014} studied the on-line bug
databases of five representative open-source software projects, as 
shown in Table \ref{tab:app_bug}. Through a mix of random sampling and 
manual inspection, they 
found 65 performance problems that are perceived and reported by users. 
Among these 65 problems, 45 problems are related to inefficient loops and 
hence are the target of the study 
here\footnote{The definition of ``loop-related'' in this paper is a little
bit broader than earlier paper~\cite{SongOOPSLA2014}, which only considers
43 problems as loop-related. }.
More details can be found in previous papers that collected
these bugs. 

\subsection{Assessment}
\label{sec:study_ob}

\comment{
\begin{table*}[tb!]
%\begin{adjustwidth}{-.5in}{-.5in}
\small
\centering
{
\begin{tabular}{|lcccccc|}
\hline
&Apache&Chrome&GCC&Mozilla&MySQL&Total\\
\hline
Total \# of loop bugs & 11 & 4 & 8 & 12 & 10 & 45 \\
\hline
\multicolumn{1}{|l}{Cross-{\bf iteration} Redundancy}
&7&1&2&1&1&12\\
\multicolumn{1}{|l}{ Cross-{\bf loop} Redundancy}
&3&0&2&2&2&9\\
\multicolumn{1}{|l}{ {\bf 0*} Resultless}
&0&0&0&0&0&0\\
\multicolumn{1}{|l}{ {\bf 0*1?} Resultless}
&0&0&0&2&3&5\\
\multicolumn{1}{|l}{{\bf [0$|$1]*} Resultless}
&0&1&1&1&1&4\\
\multicolumn{1}{|l}{{\bf 1*} Resultless}
&1&2&3&6&3&15\\
%&0&2&0&5&0&7&B(4)$|$S(3)\\
%&1&0&3&1&3&8&\\
  %MySQL15811 is moved from 1* to here; consider it as fixed by M
%\hline
%\multicolumn{8}{|c|}{ \# of {\textit {Other}} bugs}\\
%\multicolumn{1}{|l}{Not in above categories}
\hline
\end{tabular}
}
%\end{adjustwidth}
\caption{Number of bugs in each root-cause category. 
}
\label{tab:root}
\end{table*}
}



\comment{
\def\cca#1{\cellcolor{black!#10}\ifnum #1>5\color{white}\fi{#1}}
%For ranges 0-10, multiply by 10 by adding 0 after #1
\begin{table}[tb!]
%\begin{adjustwidth}{-.5in}{-.5in}
\small
\centering
{
\begin{tabular}{lcccccc}
%\hline
                                 &{\bf 0*}    & {\bf 0*1?} & {\bf [0$|$1]*}   & {\bf 1*} & {\bf C-I}  & {\bf C-L} \\
%\hline
 B                               & \cca{0}    & \cca{0}    & \cca{0}          & \cca{4}  & \cca{4}     & \cca{4}   \\
%\hline
 M                               & \cca{0}    & \cca{0}    & \cca{0}          & \cca{0}  & \cca{7}     & \cca{5}   \\
 S                               & \cca{0}    & \cca{1}    & \cca{4}          & \cca{4}  & \cca{0}     & \cca{0}   \\
 C                               & \cca{0}    & \cca{4}    & \cca{0}          & \cca{0}  & \cca{0}     & \cca{0}   \\
 O                               & \cca{0}    & \cca{0}    & \cca{0}          & \cca{7}  & \cca{1}     & \cca{0}   \\
%\hline

\end{tabular}
}
%\end{adjustwidth}
\caption{Number of bugs fixed by each strategy:
B(atching),  
M(emoization), 
S(kipping the loop),
C(hange the data structure), and O(thers). 
``C-I'': cross- iteration redundancy. 
``C-L'': cross-loop redundancy. 
}
\label{tab:correlation}
\end{table}
}


\paragraph{Coverage}
As shown in Table \ref{tab:root}, 
our taxonomy does cover all inefficient loops under study. 
Resultless loops are about as common as redundant loops
(24 vs. 21).
Not surprisingly, 0* loops
are rare in mature software. In fact, no bugs in this
benchmark suite belong to this category.
All other root-cause sub-categories are well represented.


\paragraph{Actionability}
As demonstrated in Table \ref{tab:root}, 
the root-cause categories in our taxonomy are well correlated with
fix strategies.
This indicates that our taxonomy is actionable --- once the root cause
is identified, developers roughly know how to fix the problem.
For example, 
almost all 0*1? resultless loops are fixed by data-structure changes;
all [0$|$1]* resultless loops are
fixed by conditionally skipping the loop;
almost all redundant loops are fixed either by 
memoization or batching. 
The only problem is that there are no silver bullets for fixing 1* loops.
%all inefficient loops with xx root cause are fixed by xxx.
%xxx

\paragraph{Generality}
The root-cause categories in our taxonomy are designed to be 
generic. Table \ref{tab:root} also shows that these categories 
each appears in multiple
application in our study. The only exception is $0^*$-resultless, which
never appears. 


In summary, the study above informally demonstrates that our taxonomy
is suitable to guide our design of \Tool.
%chances to satisfy the coverage and accuracy requirements of performance 
%diagnosis.
%
%TODO even for the other, there are things we could do ...

\subsection{Caveats} 
We do not intend to overly generalize the evaluation results above.
The performance bugs examined above cover a variety of applications, workload, 
development environments and programming languages. 
However, there are still uncovered cases, like distributed systems, 
and scientific computation. 

Previous work~\cite{PerfBug, SongOOPSLA2014} uses developers tagging and
on-line discussion to judge whether a bug report is about
performance problems and whether 
the performance problem under discussion is noticed and reported by users
or not.
We follow the methodology used in previous work~\cite{SongOOPSLA2014} to judge 
whether the root-cause of a performance problem is related to loops or not.
We do not intentionally ignore any aspect of loop-related performance problems. 
Some loop-related performance problems may never be noticed by end users
or never be fixed by developers, and hence skip our study. However,
there are no conceivable ways to study them. 

\comment{
We believe that the bugs in our study provide a representative sample of the 
well-documented and
fixed performance bugs that are user-perceived and loop-related in the studied 
applications. 
Since we did not set up the root-cause taxonomy to fit particular
bugs in this bug benchmark suite, we believe our taxonomy and diagnosis
framework presented below will go beyond these sampled performance bugs. 
}
