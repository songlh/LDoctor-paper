\section{Introduction}
\label{sec:intro}
\subsection{Motivation}

Performance bugs\footnote{We also refer performance bugs as performance problems
following previous works in this area~\cite{PerfBug,Alabama,SongOOPSLA2014}} 
are software implementation mistakes that cause unnecessary performance
degradation in software. They widely exist in deployed software due to the 
complexity of modern software and the lack of testing support
~\cite{s2e,PerfBug,perf.fse10,rily.perftest,perfantipattern,xiao13:context}. 
They annoy end users and waste energy during production runs, and 
have already caused highly publicized failures~\cite{ACA-health,colorado}.
Tools that can help
developers quickly and accurately diagnose performance problems,
find performance-wasting root causes, and fix performance problems
are sorely desired. 

Like general failure diagnosis, 
performance diagnosis starts from studying a problem symptom and hopefully ends
at identifying the root cause and suggesting a fix strategy. 
In the context of performance problems,
the symptom is often
an execution showing severe absolute or
relative slowness \cite{SongOOPSLA2014}; the root cause contains 
information like which code region is inefficient and why
it is inefficient. An effective diagnosis tool can help developers quickly
and correctly figure out a patch to fix the performance problem.

Also like general failure diagnosis, ideal performance diagnosis tools should
satisfy three criteria.
\begin{itemize}
\item Coverage. 
Real-world performance problems are caused by a wide variety of reasons. It 
would be good if the diagnosis tool can cover a good portion of common
root causes of real-world problems.

\item Accuracy. 
The accuracy requirement applies for two aspects of performance diagnosis: (1)
the inefficient code regions need to be accurately located and (2)
the reason a specific region is inefficient needs to be accurately
explained, so that developers can quickly fix the performance problem.

\item Performance. 
Diagnosis often requires collecting run-time information. The lower the overhead
is, the easier for the diagnosis tool to be deployed, especially for 
production-run usage scenarios. 
\end{itemize}

Although progress has been made, no existing tools can satisfy the above
three requirements.

Profiling is the state of practice in performance diagnosis.
It is far from providing the desired \textit{accuracy}, as it is designed to
tell where computation resources are spent, 
but not where and why computation resources are wasted. 
In many cases, the root-cause function of a performance problem may not even
get ranked by a profiler \cite{SongOOPSLA2014}.

Performance bug detection tools use static or dynamic analysis to identify
code regions that match specific inefficiency patterns 
\cite{Alabama,CARAMEL, Cachetor,Xu:2010:FLD:1806596.1806617,Dufour:2008:STC:1453101.1453111, Xu:2009:GFP:1542476.1542523, Xu:2010:DIC:1806596.1806616}. 
Unfortunately, these tools are not designed for performance diagnosis.
Consequently, they are not designed to provide \textit{coverage} for a wide
variety of real-world performance problems; they do not target a specific
performance symptom, and hence are at disadvantage in terms of diagnosis
\texttt{accuracy}; dynamic performance-bug detection tools often lead to 
10X slowdowns or more \cite{Cachetor,Xu:2010:FLD:1806596.1806617,Alabama}, 
not ideal in terms of \textit{performance}.

Recently, progress has been made on applying statistical debugging to 
performance diagnosis \cite{SongOOPSLA2014}. Given the symptom of 
a performance problem (e.g., two similar inputs leading to very
different execution speed),
statistical debugging, an approach widely used for diagnosing
functional failures \cite{liblit03,liblit05,tarantula1}, can
accurately identify control-flow constructs that are most correlated with
the performance problem by comparing problematic runs and regular runs.
Unfortunately, for loop-related 
performance problems, which contribute to two thirds of
real-world performance problems studied in previous work 
\cite{SongOOPSLA2014,PerfBug}, statistical debugging is not very effective.
Although it can identify the root-cause loop, it does not provide any 
information regarding why the loop is inefficient and hence is not very helpful
in fixing the performance problem.

\begin{figure}
\centering
\lstset{basicstyle=\ttfamily\fontsize{8}{8}\selectfont,
     morekeywords={+},keepspaces=true}
  \mbox{\lstinputlisting[mathescape,boxpos=t]{figures/GCC27733.c}}
\caption{A real-world performance bug in GCC (the `-' and `+' demonstrate the patch)}
\label{fig:GCC27733}
\end{figure}

Figure~\ref{fig:GCC27733} shows a real-world performance bug in GCC. 
Function \texttt{synth\_mult} is used to compute the best algorithm for 
multiplying \texttt{t}. It is so time consuming that developers tried 
speeding it up through memoization ---
hash-table
\texttt{alg\_hash} is used to remember which \texttt{t} has been processed
in the past and what is the processing result. 
Unfortunately, a small mistake in the type declaration of hash-table
entry \texttt{alg\_hash\_entry} causes memoization not to work when \texttt{t}
is larger than the maximum value type-\texttt{int} can represent.
As a result, under certain workload, \texttt{synth\_mult} conducts a lot of
redundant computation for same \texttt{t} values recursively, leading to
huge performance problems.  In practice, developers know the problem is inside
\texttt{synth\_mult} very early. However, since this function is 
complicated, it took them several weeks to figure out the root cause. 
If a tool can tell them not only which loop\footnote{Recursive functions are
handled similarly as loops in this paper.}
is the root cause but also why a loop is inefficient (i.e., lot of redundant
computation and the need for better memoization in this example), 
the diagnosis and bug fixing
process would be much easier for developers. 

Clearly, more research is needed to improve the state of the art of performance
diagnosis. Diagnosis techniques that can provide better
\textit{coverage} and \textit{accuracy} for common performance problems,
especially loop-related performance problems, with good
performance is well desired.


\subsection{Contributions}
This paper presents a tool, \Tool, that can help effectively diagnose
inefficient loop problems with good coverage, accuracy, and performance.

\Tool targets the most common type of performance problems, inefficient
loops, according to previous
empirical studies \cite{SongOOPSLA2014,PerfBug}.
\Tool also targets a challenging problem. Although
statistical debugging \cite{SongOOPSLA2014}
can help identify suspicious loops that are highly correlated with 
user-perceived performance problems,
it provides no information about why the suspicious loop is inefficient,
not to mention providing fix suggestions. Although various
bug-detection techniques \cite{Cachetor,Alabama,CARAMEL} can help detect
loop-related performance bugs, they all suffer from generality problems,
covering only a specific type of loop problems each; the dynamic
tools \cite{Cachetor,Alabama} also suffer from performance
problem, imposing 10X slowdown or more.

\Tool tackles this challenging problem in three steps.
%}

First, figuring out a taxonomy for the root causes of common inefficient loops.
Such a taxonomy is the prerequisite to developing a general and accurate
diagnosis tool. Guided by a thorough study of 45 real-world inefficient
loop problems, we come up with a hierarchical root-cause taxonomy for
inefficient loops. 
%Specifically, we consider \textit{resultless} and
%\textit{redundancy} as two main categories of root causes for inefficient
%loops, representing loops that spend effort without producing
%any results or producing redundant results. They are further divided into
%sub-categories based on the granularity of the resultless or redundant work. 
Our empirical study shows that our taxonomy is general enough to cover
common inefficient loops, and also specific enough to help developers understand
and fix the performance problem. We will present more details 
in Section \ref{sec:study}.
 
Second, building a tool(kit) \Tool that can automatically and accurately
identify the fine-grained root-cause type of a suspicious loop, 
together fix-strategy suggestions. 
We achieve this following several principles:

\begin{itemize}
\item Focused checking. 
Different from performance-bug detection tools that blindly checks the whole
software, \Tool focuses on performance symptoms and only check inefficient
loop candidates identified by existing tools, a statistical debugging
tool in our current prototype \cite{SongOOPSLA2014}. 
This focused study is crucial for \Tool to achieve high
accuracy.

\item Root-cause taxonomy guided design. To provide the needed coverage, we follow
the root-cause taxonomy discussed above and design analysis routines for 
every root-cause sub-category. Given a candidate inefficient loop, we will 
apply a series of analysis to it to see if it matches any type of inefficiency.
%We will also put the analysis results from different analysis routines together
%to figure out the most likely root cause category for an inefficient loop.

\item Static-dynamic hybrid analysis to balance performance and accuracy.
As we will see, static analysis alone cannot accurately identify 
inefficiency root causes, especially because some inefficiency problems only
happen under specific workload. However, pure dynamic analysis will 
cause too large 
run-time overhead. Therefore, we take a hybrid approach to achieve both
performance and accuracy goals.
\end{itemize}

Third, using sampling to further lower the run-time overhead of \Tool, without
degrading diagnosis capability. Random sampling is a natural fit for performance
diagnosis due to the repetitive nature of inefficient code, especially
inefficient loops.

We have evaluated \Tool on 18 real-world performance problems. 
Evaluation results show that \Tool can accurately identify the detailed
root cause for all benchmarks and provide correct fix-strategy suggestion for
16 benchmarks. All these are achieved with low run-time overhead.
