\section{Related Works}
\label{sec:related}

Profilers are the most commonly used tools that help developers understand
performance
\cite{Zaparanuks:2012:AP:2254064.2254074,Coppa:2012:IP:2254064.2254076,
D'Elia:2011:MHC:1993498.1993559, 
Mytkowicz:2010:EAJ:1806596.1806618,
Ravindranath:2012:AMA:2387880.2387891,
Jovic:2011:CMY:2048066.2048081,IntroPerf}.
They aim to tell where 
computation resources are spent, 
not where and why computation resources are wasted. 
The root-cause code region of a performance problem often is not inside
the top-ranked function in the profiling result \cite{SongOOPSLA2014}.
Even if it is, developers still need to spend a lot of effort to understand
whether and what kind of computation inefficiency exists.

Recently proposed tools go beyond profiling.
They can identify slow call-stack patterns inside event handlers
\cite{Han:2012:PDL:2337223.2337241}, help understand performance causality
relationship among system components
\cite{TaoAsplos2014}, identify inputs 
or configuration entries that are most responsible for performance problems
\cite{Attariyan:2012:XAR:2387880.2387910},
and estimate the performance impact of any
potential optimization at any point of a multi-threaded program
\cite{coz.sosp15}. 
They all target different problems from \Tool. 
\comment{
\Tool targets 
inefficient loops, a common
inefficiency problem that contributes to
about two thirds of real-world user-perceived performance problems studied
by previous work \cite{SongOOPSLA2014}, and tries to pin-point the exact
root cause and provide fix-strategy suggestions.
}

\Tool is most related to
the recent statistical performance debugging work
\citep{SongOOPSLA2014}, both trying to identify source-code level root causes
for user-perceived performance problems. 
This previous work identifies which loop is most correlated with 
a performance symptom through statistical analysis, 
but cannot answer whether or 
what type of inefficiency this loop contains.
\Tool complements it by
accurately pointing out whether the suspicious loop is inefficient,
which type of inefficiency a loop contains if any, and 
what is the best fix strategy.

Many dynamic and static analysis tools have been built to detect different
types of performance problems, such as 
run-time bloat~\cite{Dufour:2008:STC:1453101.1453111, Xu:2009:GFP:1542476.1542523, Xu:2010:DIC:1806596.1806616}, 
low-utility data structures~\cite{Xu:2010:FLD:1806596.1806617}, 
cacheable data~\cite{Cachetor}, false sharing in
multi-thread programs~\cite{Liu:2011:SPD:2048066.2048070},
inefficient nested loops~\cite{Alabama},
loops with unnecessary iterations~\cite{CARAMEL,IsilDillig.PLDI15},
input-dependent loops~\cite{xiao13:context}. 
%Some static tools \cite{PerfBug} identify performance defects that violate
%specific efficiency rules, often related to API usage. 

As discussed in 
Section \ref{sec:intro}, these tools are all useful, 
but are not suitable for performance diagnosis.
Bug detectors are not guided by any specific
performance symptoms. Consequently, they take different coverage-accuracy
trade-offs from \Tool. \Tool tries to cover a wide variety of root-cause 
categories and is more aggressive in identifying root-cause categories, because
it is only applied to a few loops known to be highly
correlated with a specific performance symptom.
Performance-bug detection tools are more conservative and try hard
to lower false positive rates, because they need to process the
whole program, instead of just a few loops. This requirement causes bug
detectors to each focus on a specific root-cause category.
%In fact, most inefficient loops in our study are very difficult to be detected
%by a generic performance bug detectors --- it is very difficult to prove them to
%be inefficient without application-specific domain knowledge.
%For example, only a couple of inefficient loops out of the 45 studied
%in Table \ref{tab:root} can be detected by recent static inefficient
%loop checkers
%\cite{CARAMEL,IsilDillig.PLDI15}.
Bug detectors also do not aim to provide fix suggestions.
For the few that do provide \cite{CARAMEL}, they only focus on very specific
fix patterns, such as adding a break into the loop. 
In addition, dynamic performance bug detectors often lead to 10X slowdowns or 
more, and never tried sampling.

Automated tools have been developed to detect inefficient loop bugs
\cite{Alabama, CARAMEL, IsilDillig.PLDI15}. 
As discussed above, these tools are good detection tools but not suitable for 
diagnosis. 
For example, Toddler~\cite{Alabama} only targets inefficient nested loops, and
hence can only cover about half of the bugs in our \emph{general} benchmark suite.
Being a dynamic tool, it also incurs 10X or more slowdowns, which is much more
expensive than \Tool. 
Caramel~\cite{CARAMEL} statically detects inefficient loops that can be fixed
by adding conditional-breaks. Very few bugs in Table \ref{tab:root} are like this.
CLARITY~\cite{IsilDillig.PLDI15} statically detects redundant traversal bugs, 
which arise if a program fragment repeatedly iterates over a data structure, such as an array or list, that has not been modified between successive traversals of the data structure.
Like Toddler, it targets nested loops.
We expect it to detect a sub-set of the cross-loop redundancy
loops and a sub-set of 0*1? resultless loops in our benchmark set.

There are many test inputs generation techniques that help
performance testing 
\cite{WISE, EventBreak, SpeedGun}.
Some techniques aim to improve the test selection or prioritization during 
performance testing~\cite{Forepost,Huang:2014:PRT:2568225.2568232}.  
All these techniques combat performance bugs from different aspects from performance diagnosis.
