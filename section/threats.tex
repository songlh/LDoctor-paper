\section{Threats to Validity}
\label{sec:threats}

Of course, our taxonomy does not cover \emph{all} loop inefficiency problems.
For example, some loops may be vulnerable to false sharing problems or lock
contention problems, which are out of the scope of our taxonomy. 


The static program analysis and optimization.
This analysis is complete, but not sound. We may identify side-effect
instructions that are actually not (Section \ref{sec:s_workless}).
The optimization discussed in Section \ref{sec:perf}.

The resuleless and redundancy rate thresholds could lead to false positives
and false negatives.

Our evaluation.
What presented above reflects our best effort in assessing our root-cause
taxonomy using a non-biased set of real-world inefficient loop problems that
have been perceived by users and fixed by developers.

The numbers presented above should be interpreted together with our methodology
and should not be overly generalized.
The performance bugs examined above cover a variety of applications, workload, 
development environments and programming languages. 
However, there are definitely uncovered cases, like problems in distributed systems
and scientific computing systems, and others. 

Previous work~\cite{PerfBug, SongOOPSLA2014} uses developers tagging and
on-line discussion to judge whether a bug report is about
performance problems and whether 
the performance problem under discussion is noticed and reported by users
or not.
We do not intentionally ignore any aspect of loop-related performance problems. 
Some loop-related performance problems may never be noticed by end users
or never be fixed by developers, and hence skip our study. However,
there are no conceivable ways to study them. 

\comment{
We believe that the bugs in our study provide a representative sample of the 
well-documented and
fixed performance bugs that are user-perceived and loop-related in the studied 
applications. 
Since we did not set up the root-cause taxonomy to fit particular
bugs in this bug benchmark suite, we believe our taxonomy and diagnosis
framework presented below will go beyond these sampled performance bugs. 
}
